%!TEX root = ./icml_2019_sketchmcmc.tex

\section{Regularized Sliced-Wasserstein Flows for Generative Modeling}





\subsection{Construction of the gradient flow}

% \vspace{-5pt}


% \textbf{Construction of the flow: }
%

In this paper, we propose the following functional minimization problem on $\PS_2(\Omega)$ for implicit generative modeling:
% We propose in this paper to consider the minimization of the functional $\F^{\nu}_{\lambda}$ on $\PS_2(\Omega)$, that is defined as follows:
\begin{equation}
 \min_{\mu} \Bigl\{ \F^{\nu}_\lambda(\mu) \triangleq  \frac1{2} \SW^2(\mu, \nu) + \lambda \He(\mu) \Bigr\},  \label{eqn:sw_optim}
\end{equation}
where $\lambda >0$ is a regularization parameter and $\He$ denotes the negative entropy defined by $\He(\mu) \triangleq \int_{\Omega} \rho(x) \log \rho(x) dx $ if $\mu$ has density $\rho$ with respect to the Lebesgue measure and $\He(\mu) = + \infty$ otherwise. Note that the case $\lambda =0$ has been already proposed and studied in \cite{bonnotte2013unidimensional} in a more general OT context. Here, in order to introduce the necessary noise inherent to generative model, we suggest to penalize the slice-Wasserstein distance using $\He$. In other words, the main idea is to find a measure $\mu^\star$ that is close to $\nu$ as much as possible and also has a certain amount of entropy to make sure that it is sufficiently expressive for generative modeling purposes.
The importance of the entropy regularization becomes prominent in practical applications where we have finitely many data samples that are assumed to be drawn from $\nu$. In such a circumstance, the regularization would prevent $\mu^\star$ to collapse on the data points and therefore avoid `over-fitting' to the data distribution. Note that this regularization is fundamentally different than the one used in Sinkhorn distances \cite{genevay2018learning}.


In our first result, we show that there exists a flow $(\mu_t)_{t\geq0}$ in $(\PS(\cB(0,r)),\W)$ which decreases along $\F_\lambda^\nu$, where $\cB(0,a)$ denotes the closed unit ball centered at $0$ and radius $a$. This flow will be referred to as a generalized minimizing movement scheme (see Definition~$1$ in \supp).  In addition, the flow $(\mu_t)_{t \geq 0}$ admits a density $\rho_t$ with respect to the Lebesgue measure for all $t>0$ and $(\rho_t)_{t \geq 0}$ is solution of a non-linear PDE (in the weak sense). %  
%
%
\begin{thm}
\label{thm:continuity}
Let $\nu$ be a probability measure on $\cB(0,1)$ with a strictly positive smooth density. Choose a regularization constant $\lambda > 0$ and radius $r > \sqrt{d}$. Assume that $\mu_0 \in \mathcal{P}(\cB(0,r))$ is absolutely continuous with respect to the Lebesgue measure with density $\rho_0 \in \mrl^{\infty}(\cB(0,r))$. There exists a generalized minimizing movement scheme  $(\mu_t)_{t \geq 0}$ given by Theorem~S$2$ in \supp and if $\rho_t$ stands for the density of $\mu_t$ for all $t \geq 0$, then $(\rho_t)_t$ satisfies the following continuity equation:
\begin{align}
\frac{\partial \rho_t}{\partial t}   &= -\divop (v_t \rho_t) + \lambda \Delta \rho_t, \\
 % v_t(x) \triangleq v(x,\mu_t) &= - \int_{\Sp^{d-1}} \psi_{t, \theta}'(\langle x , \theta \rangle ) \theta d\theta  \label{eqn:gradflow_reg}
% \end{align}
% \begin{align}
v_t(x) \triangleq v(x,\mu_t) &= - \int_{\Sp^{d-1}} \psi_{t, \theta}'(\langle x , \theta \rangle ) \theta d\theta  \label{eqn:gradflow_reg}
\end{align}
in a weak sense. Here, $\Delta$ denotes the Laplacian operator, $\divop$ the divergence operator, and $\psi_{t,\theta}$ denotes the Kantorovich potential between $\theta^*_{\#}\mu_t$ and $\theta^*_{\#}\nu$.
\end{thm}
The precise statement of this Theorem, related results and its proof are postponed to \supp. For its proof, we use the technique introduced in \cite{jordan1998variational}: we first prove the existence of a generalized minimizing movement scheme by showing that the solution curve $(\mu_t)_t$ is a limit of the solution of a time-discretized problem. Then we prove that the curve $(\rho_t)_t$ solves the PDE given in \eqref{eqn:gradflow_reg}.





\subsection{Connection with stochastic differential equations}

% \textbf{Connection with stochastic differential equations: }
%
% 
As a consequence of the entropy regularization, we obtain the Laplacian operator $\Delta$ in the PDE given in \eqref{eqn:gradflow_reg}. We therefore observe that the overall PDE is a Fokker-Planck-type equation \cite{bogachev2015fokker} that has a well-known probabilistic counterpart, which can be expressed as a stochastic differential equation (SDE). More precisely, let us consider a stochastic process $(X_t)_{t}$, that is the solution of the following SDE starting at $X_0 \sim \mu_0$:
\begin{align}
d X_t = v(X_t,\mu_t) dt + \sqrt{2 \lambda } d W_t, \label{eqn:sde}
\end{align}
where $(W_t)_t$ denotes a standard Brownian motion. Then, the probability distribution of $X_t$ at time $t$ solves the PDE given in \eqref{eqn:gradflow_reg}. This informally means that, if we could simulate \eqref{eqn:sde}, then the distribution of $X_t$ would converge to the solution of \eqref{eqn:sw_optim}, therefore, we could use the sample paths $(X_t)_t$ as samples drawn from $(\mu_t)_t$. However, in practice this is not possible due to two reasons: (i) the drift $v_t$ cannot be computed analytically since it depends on the probability distribution of $X_t$, (ii) the SDE \eqref{eqn:sde} is a continuous-time process, it needs to be discretized.








We now focus on the first issue.
% 
We observe that the SDE \eqref{eqn:sde} is similar to McKean-Vlasov SDEs \cite{veretennikov2006ergodic,mishura2016existence}, a family of SDEs whose drift depends on the distribution of $X_t$. By using this connection, we can borrow tools from the relevant SDE literature \cite{malrieu03,cgm-08} for developing an approximate simulation method for \eqref{eqn:sde}.

Our approach is based on defining a \emph{particle system} that serves as an approximation to the original SDE \eqref{eqn:sde}. The particle system can be written as a collection of SDEs, given as follows \cite{bossy1997stochastic}:
\begin{align}
d X_t^i = v(X_t^i, \mu_t^{N}) dt + \sqrt{2 \lambda } d W_t^i \> , \quad i = 1,\dots, N, \label{eqn:sde_particle}
\end{align}
where $i$ denotes the particle index, $N \in \mathbb{N}_+$ denotes the total number of particles, and $\mu_t^N = (1/N) \sum_{j=1}^N \delta_{X_t^j}$ denotes the empirical distribution of the particles $\{X_t^j\}_{j=1}^N$. This particle system is particularly interesting, since (i) one typically has $\lim_{N \rightarrow \infty} \mu_t^{N}= \mu_t $ with a rate of convergence of order ${\cal O}(1/\sqrt{N})$ for all $t$ \cite{malrieu03,cgm-08}, and (ii) each of the particle systems in \eqref{eqn:sde_particle} can be simulated by using an Euler-Maruyama discretization scheme. We note that the existing theoretical results in \cite{veretennikov2006ergodic,mishura2016existence} do not directly apply to our case due to the non-standard form of our drift. However, we conjecture that a similar result holds for our problem as well. Proving such a result would be very involved and it is out of the scope of this study. 

\subsection{Approximate Euler-Maruyama discretization}
% \textbf{Approximate Euler-Maruyama discretization:}
%
In order to be able to simulate the particle SDEs \eqref{eqn:sde_particle} in practice, we propose an approximate Euler-Maruyama discretization for each particle SDE.
The algorithm iteratively applies the following update equation: ($\forall i \in  \{1,\dots,N\}$)
% 
\begin{align}
\bar{X}^i_0 \simiid \mu_0, \>\> \bar{X}^i_{k+1} = \bar{X}^i_k + h \hspace{0.5pt} \hat{v}_k(\bar{X}^i_k) + \sqrt{2 \lambda h} Z^i_{k+1}, \label{eqn:euler_particle}
\end{align}
where $k \in \mathbb{N}_+$ denotes the iteration number, $Z^i_k$ is a standard Gaussian random vector in $\R^d$, $h$ denotes the step-size, and $\hat{v}_k$ is a short-hand notation for a computationally tractable estimator of the original drift $v(\cdot, \bar{\mu}_{kh}^N)$, with $\bar{\mu}_{kh}^{N} = (1/N) \sum_{j=1}^N \delta_{\bar{X}_k^j}$ being the empirical distribution of $\{\bar{X}_k^j\}_{j=1}^N$. A question of fundamental practical importance is how to compute this function $\hat{v}$.

We propose to approximate the integral in \eqref{eqn:gradflow_reg} via a simple Monte Carlo estimate.
% 
At each iteration $k$, this is done by drawing $N_\theta$ uniform i.i.d.\ samples from the sphere $\Sp^{d-1}$, $\{\theta_{k,n}\}_{n=1}^{N_\theta}$, and computing:
%
\begin{align}
\hat{v}_k(x) \triangleq - (1/{N_\theta}) \sum\nolimits_{n=1}^{N_\theta} \psi_{k, \theta_{k,n}}'(\langle\theta_{k,n},x\rangle ) \theta_{k,n}, \label{eqn:approxdrift}
\end{align}
%
where for any $\theta$, $\psi_{k, \theta}'$ is the derivative of the Kantorovich potential (cf.\ Section~\ref{sec:techbg}) that is applied to the OT problem from $\theta^*_\#\bar{\mu}_{kh}^{N}$ to $\theta^*_\#\nu$: i.e.\,
\begin{align}
   \psi_{k, \theta}'(z) = \bigl[ z - (F^{-1}_{\theta^*_\#\nu} \circ F_{\theta^*_\#\bar{\mu}_{kh}^{N}}) (z)  \bigr] .%, \qquad \forall z \in \R$.
 \end{align}
%where $\{\theta_n\}_{n=1}^{N_\theta}$ denotes a collection of i.i.d.\ samples that are drawn uniformly on $\Sp^{d-1}$, $F_{\theta^*_{n\#}\mu}$ and $F_{\theta^*_{n\#}\nu}$ denote the (one-dimensional) CDFs of the distributions $\theta^*_{n\#}\mu$ and $\theta^*_{n\#}\nu$, respectively. In \eqref{eqn:approxdrift} we used the definition of the derivative of the one-dimensional Kantorovich potential, which was defined in Section~\ref{sec:sw}.

% \begin{wrapfigure}{R}{0.52\textwidth}
% % \vspace{-14pt}
% \vspace{-10pt}
%     \begin{minipage}{0.52\textwidth}
     \begin{algorithm2e}[t]
		 \SetInd{0.1ex}{1.5ex}
		 \DontPrintSemicolon
		 \SetKwInOut{Input}{input}
		 \SetKwInOut{Output}{output}
		 \Input{${\cal D} \equiv \{y_i\}_{i=1}^P$, $\mu_0$, $N$, $N_\theta$, $h$, $\lambda$}
		 \Output{$\{\bar{X}_K^i\}_{i=1}^N$}
		 {\color{purple} \small \tcp{Initialize the particles}}
		 $\bar{X}_0^i \simiid \mu_0$, \hfill $i = 1,\dots,N$ \\
		 \For{$k = 0,\dots K-1$}
     {
		    {\color{purple} \small \tcp{Generate random directions}}
		    $\theta_{k,n} \sim \mathrm{Uniform}(\Sp^{d-1})$, \hfill $n = 1,\dots,N_\theta$\\
        \For{$\theta\in\{\theta_{k,n}\}_{n=1}^{N_\theta}$}
        {
        {\color{purple} \small \tcp{CDF of projected particles}}
        $F_{\theta^*_\#\bar{\mu}_{kh}^{N}}=\textnormal{CDF}\{\langle\theta,\bar{X}_k^i \rangle\}_{i=1}^N$\\
        {\color{purple} \small \tcp{Quantiles of projected target}}
        $F^{-1}_{\theta^*_\#\nu}=\textnormal{QF}\{\langle\theta,y_i \rangle\}_{i=1}^P$\\
          %$\theta^*_{n\#}\bar{\mu}^N_{kh} = \frac1{N} \sum_{i=1}^N \delta_{\langle \theta_n, \bar{X}_{k}^i \rangle}$, \hfill $n = 1,\dots,N_\theta$ \\
          %$\theta^*_{n\#}\nu = \frac1{N} \sum_{i=1}^P \delta_{\langle \theta_n, y_i \rangle} $, \hfill $n = 1,\dots,N_\theta$ \\
        }
        {\color{purple} \small \tcp{Update the particles}}
        $\bar{X}_{k+1}^i = \bar{X}_{k}^i - h \hspace{0.5pt} \hat{v}_k(\bar{X}^i_k) + \sqrt{2 \lambda h} Z^i_{k+1}$ \vspace{2pt} \\
        $\hfill i = 1,\dots,N$

		 }
		 \caption{Sliced-Wasserstein Flow (SWF)}
		 \label{algo:flow}
	 \end{algorithm2e}
% \end{minipage}
% % \vspace{-20pt}
% \vspace{-10pt}
% \end{wrapfigure}

% \newcommand{\shr}{{\lserif\#}}
For any particular $\theta\in\Sp^{d-1}$, the QF, $F_{\theta^*_\#\nu}^{-1}$ for the projection of the target distribution $\nu$ on $\theta$ can be easily computed from the data. This is done by first computing the projections $\langle \theta, y_i\rangle$ for all data points $y_i$, and then computing the empirical quantile function for this set of $P$ scalars.
%
Similarly, $F_{\theta^*_\#\bar{\mu}_{kh}^{N}}$, the CDF of the particles at iteration $k$, is easy to compute: we first project all particles $\bar{X}_k^i$ to get $\langle \theta, \bar{X}_k^i\rangle$, and then compute the empirical CDF of this set of $N$ scalar values.

In both cases, the true CDF and quantile functions are approximated as a linear interpolation between a set of the computed $Q\in\mathbb{N}_+$ empirical quantiles.
Another source of approximation here comes from the fact that the target $\nu$ will in practice be a collection of Dirac measures on the observations $y_i$. Since it is currently common to have a very large datasets, we believe this approximation to be accurate in practice for the target.

Even though the error induced by these approximation schemes can be incorporated into our current analysis framework, we choose to neglect it for now, because (i) all of these one-dimensional computations can be done very accurately and (ii) the quantization of the empirical CDF and QF can be modeled as additive Gaussian noise that enters our discretization scheme \eqref{eqn:euler_particle} \cite{van1998asymptotic}. Therefore, we will assume that $\hat{v}_k$ is an \emph{unbiased} estimator of $v$, i.e.\ $\E[\hat{v}(x,\mu)] = v(x,\mu)$, for any $x$ and $\mu$, where the expectation is taken over $\theta_{k,n}$.



The overall algorithm is illustrated in Algorithm~\ref{algo:flow}. It is remarkable that the updates of the particles only involves the learning data $\{y_i\}$ through the CDFs of its projections on the many $\theta_{k,n}\in\Sp^{d-1}$. This has a fundamental consequence of high practical interest: these CDF may be computed in a massively distributed manner that is independent of the sliced Wasserstein flow. This aspect is reminiscent of the \textit{compressive learning} methodology \cite{gribonval2017compressive}, except we exploit quantiles of random projections here, instead of random moments as done there.



Besides, we can obtain further reductions in the computing time if the CDF, $F_{\theta^*_\#\nu}$ for the target is computed on random mini-batches of the data, instead of the whole dataset of size $P$. This simplified procedure might also have some interesting consequences in privacy-preserving settings: since we can vary the number of projection directions $N_\theta$ for a each data point $y_i$, by using the compressed sensing theory \cite{donoho2009observed}, we may guarantee that $y_i$ cannot be recovered via these projections, by simply picking fewer projections than necessary for reconstruction.




% \begin{remark}
We also note that, even though the proposed algorithm assumes that $\mu$ and $\nu$ have the same dimensionality, the theory in fact allows us to use a lower dimensional $\mu$ by using a simple procedure: if $\mu$ is $d' \ll d$ dimensional, we can generate a random sample from $\mu$ and multiply it by any full-rank matrix of size $d \times d'$, and then use this sample as the input to the algorithm. We have the same theoretical guarantees with this procedure. 
%
%
% \umut{True. Don't we discuss this in the conclusion?}
%
% \ubul{8} In fact, theory allows us to use a lower dimensional $\mu$ by using
% % the only requirement on $\mu$ is that it has finite second-order moments and is non-atomic.
% a simple trick: if $\mu$ is $d' \ll d$ dimensional, we can generate a random sample from $\mu$ and multiply it by any full-rank matrix of size $d' \times d$, using this output in the algorithm. We did not notice any difference in performance between using this trick for small input dimensions and directly taking $\mu$ as high dimensional, hence we omitted this development, but will include it again.
% \end{remark}



%%% Local Variables:
%%% mode: latex
%%% TeX-master: "aistats_2019_sketchmcmc"
%%% End:
