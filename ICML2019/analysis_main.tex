
\subsection{Finite-time analysis for the infinite particle regime}
In this section we will analyze the behavior of the proposed algorithm in the asymptotic regime where the number of particles $N \rightarrow \infty$. Within this regime, we will assume that the original SDE \eqref{eqn:sde} can be directly simulated by using an approximate Euler-Maruyama scheme, defined starting at $\bar{X}_0 \simiid \mu_0$ as follows:
\begin{align}
 \bar{X}_{k+1} = \bar{X}_k + h \hspace{0.5pt} \hat{v}(\bar{X}^i_k, \bar{\mu}_{kh} ) + \sqrt{2 \lambda h} Z_{k+1}, \qquad \label{eqn:euler_asymp}
\end{align}
where $\mub_{kh}$ denotes the law of $\bar{X}_k$ with step size $h$ and $\{Z_k\}_{k}$ denotes a collection of standard Gaussian random variables. Apart from its theoretical significance, this scheme is also practically relevant, since one would expect that it captures the behavior of the particle method \eqref{eqn:euler_particle} with large number of particles. 

In practice, we would like to approximate the measure sequence $(\mu_t)_t$ as accurate as possible, where $\mu_t$ denotes the law of $X_t$. Therefore, we are interested in analyzing the distance $\| \mub_{Kh} - \mu_{T} \|_{\TV}$, where $K$ denotes the total number of iterations, $T=Kh$ is called the horizon, and $\|\mu-\nu\|_{\TV}$ denotes the total variation distance between two probability measures $\mu$ and $\nu$: 
$\|\mu-\nu\|_{\TV}\triangleq \sup_{A \in {\cal B}(\Omega)} |\mu(A) -\nu(A) |$.
 

In order to analyze this distance, we exploit the algorithmic similarities between \eqref{eqn:euler_asymp} and the stochastic gradient Langevin dynamics (SGLD) algorithm \cite{WelTeh2011a}, which is a Bayesian posterior sampling method having a completely different goal, and is obtained as a discretization of an SDE whose drift has a much simpler form. We then bound the distance by extending the recent results on SGLD \cite{raginsky17a} to time- and measure-dependent drifts, that are of our interest in the paper.

We now present our second main theoretical result. We present all our assumptions and the explicit forms of the constants in \supp. 
\begin{thm}
\label{thm:euler}
Assume that the conditions given in \supp{} hold. Then, the following bound holds for $T=Kh$:
\begin{align}
\nonumber \| \mub_{Kh} - \mu_T \|_{\TV}^2 \leq \delta_\lambda \Biggl\{  \frac{L^2 K}{2\lambda} \Bigl( \frac{C_1 h^3}{3} + 3 \lambda d h^2 \Bigr) \hspace{13pt} \\ + \frac{C_2  \delta K h}{4\lambda} \Biggr\}	,
\end{align} 
for some $C_1,C_2,L >0$, $\delta \in (0,1)$, and $\delta_\lambda >1$.  %
\end{thm}
Here, the constants $C_1$, $C_2$, $L$ are related to the regularity and smoothness of the functions $v$ and $\hat{v}$; $\delta$ is directly proportional to the variance of $\hat{v}$, and $\delta_\lambda$ is inversely proportional to $\lambda$. The theorem shows that 
if we choose $h$ small enough, we can have a non-asymptotic error guarantee, which is formally shown in the following corollary. 
\begin{cor}
  \label{coro:precision}
  Assume that the conditions of Theorem~\ref{thm:euler} hold. Then for all $\varepsilon >0$, $K \in \mathbb{N}_+$, setting
  \begin{align}
h = (3/C_1)\wedge\left(\frac{2 \varepsilon^2 \lambda}{\delta_\lambda L^2 T}(1+3\lambda d)^{-1}\right)^{1/2}, %
  \end{align}
  we have
  \begin{align}
    \| \mub_{Kh} - \mu_T \|_{\TV} \leq \varepsilon + \left(\frac{C_2 \delta_\lambda \delta T}{4\lambda}\right)^{1/2} 
  \end{align}
  for $T=Kh$.
\end{cor}
This corollary shows that for a large horizon $T$, the approximate drift $\hat{v}$ should have a small variance in order to obtain accurate estimations. This result is similar to \cite{raginsky17a} and \cite{nguyen2019non}: for small $\varepsilon$ the variance of the approximate drift should be small as well. On the other hand, we observe that the error decreases as $\lambda$ increases. This behavior is expected since for large $\lambda$, the Brownian term in \eqref{eqn:sde} dominates the drift, which makes the simulation easier.

We note that these results establish the explicit dependency of the error with respect to the algorithm parameters (e.g. step-size, gradient noise) for a fixed number of iterations, rather than explaining the asymptotic behavior of the algorithm when $K$ goes to infinity.

