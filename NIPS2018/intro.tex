%!TEX root = ./nips_2018_sketchmcmc.tex

\section{Introduction}

% \begin{itemize}
% \item Short intro to implicit generative models
% \item Connection with optimal transport \cite{genevay2017gan}, several other papers
% \item Information about sliced-Wasserstein distance and usage in generative modeling \cite{bonnotte2013unidimensional,kolouri2018sliced,wu2017generative}
% \item Contributions of the current paper
% \begin{itemize}
% \item Development of a novel gradient flow for generative modeling purpose
% \item Establish the connections with SDEs
% \item Develop a practical way to simulate the SDE
% \item Establish theoretical guarantees
% \item Experimental validation
% \end{itemize}
% \end{itemize}


Implicit generative modeling has become very popular recently and has proven successful in various fields. Variational auto-encoders (VAE) and generative adversarial networks (GAN) are the two well-known examples. Recently, it has been shown that these methods are trying to solve an optimal transport problem.


The goal in these approaches can be briefly described as learning the underlying probability measure of a given data sample, denoted as $\nu \in \PS(\Omega_\nu)$, without assuming any explicit probability laws. The basic idea in these strategies is to start from a simple distribution denoted as $\mu \in \PS(\Omega_\mu)$ and find a measurable map $T: \Omega_\mu \mapsto \Omega_\nu$ such that the law of the output of this map will coincide with the target measure $\nu$. 

This can be explained using some `push-forward operators': we seek a mapping $T$ that `pushes $\mu$ onto $\nu$', and is formally defined as $\int_A \nu(dx) = \int_{T^{-1}(A)} \mu(dx) $ for all Borel $A \subset {\cal B}(\Omega_\nu)$. If this relation holds, we denote the push-forward operator $T_\#$, such that $T_\# \mu = \nu$.

Most of the current generative modeling strategies consider an operator that belongs to a parametric family $T_{\phi}$ with $\phi \in \Phi$, and aims to find the best parameter $\phi^\star$ that would give $T_{\phi^\star \#}\mu \approx \nu$. This is typically achieved by attempting to minimize the following optimization problem:
\begin{align}
\phi^\star = \argmin_{\phi \in \Phi} \W(T_{\phi \#}\mu, \nu),
\end{align}
where $\W$ denotes the Wasserstein distance that will be properly defined in Section~\ref{sec:techbg}. Genevay et al.\ \cite{genevay2017gan} showed that Wasserstein GANs and VAEs both use this formulation with different parametrizations. 

In this study, we follow a completely different approach and we consider a functional optimization problem, that is defined through gradient flows for measures in the Wasserstein space. Informally, the flow will have a shape as follows:
\begin{align}
\partial_t \mu_t = - \nabla_{\W} \Bigl\{ \mathrm{Cost}(\mu_t, \nu) + \mathrm{Reg}(\mu_t)\Bigr\} \label{eqn:gradflow}
\end{align}
where the functional $\mathrm{Cost}$ computes a discrepancy between $\mu_t$ and $\nu$, $\mathrm{Reg}$ denotes a regularization functional, and $\nabla_{\W}$ denotes a notion of gradient with respect to a probability density function in the $\W$ metric for probability measures\footnote{This gradient flow is similar to the the usual Euclidean gradient flows, i.e.\ $dx/dt = - \nabla (f(x) + r(x))$, where $f$ is typically the data-dependent cost function and $r$ is a regularization term. The (explicit) Euler discretization of this flow results in the gradient descent algorithm.}. If this flow can be simulated, one would hope for $\mu_t$ to converge to the minimum of the functional optimization problem: $\min_\mu ( \mathrm{Cost}(\mu, \nu) + \mathrm{Reg}(\mu)$).

However, in general these flows cannot be either solved or simulated perfectly. \umut{discuss $\SW$.}
