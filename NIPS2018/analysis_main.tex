%!TEX root = ./nips_2018_sketchmcmc.tex

\subsection{Non-asymptotic analysis}

Proposition~\ref{prop:dist_statmeas} shows that if we could simulate \eqref{eqn:sde}, then we could use the sample paths $(X_t)_t$ as
samples drawn from $\nu_\lambda$, which is not very far from $\nu$. However, this is not possible since the drift $v_t$ cannot be computed analytically, and the SDE \eqref{eqn:sde} is a continuous-time process.

We now consider the approximate Euler-Maruyama discretization, that is given as follows:
\begin{align}
\bar{X}_{k+1} = \bar{X}_k - h \hat{v}_k(\bar{X}_k) + \sqrt{2 \lambda h} Z_{k+1},
\end{align}
where $k \in \mathbb{N}_+$ denotes the iteration number, $\{Z_k\}_{k}$ denotes a series of standard Gaussian random variables, $h$ denotes the step-size, and $\hat{v}_k$ is a computable unbiased estimator of $v_{kh}$.


By Theorem 5.6.1 of \cite{bonnotte2013unidimensional}, we know that $\Psi_t$ is Lipschitz continuous. We consider the following assumptions:
\begin{assumption}
\label{asmp:lipschitz}
The drift is Lipschitz continuous, i.e.\ there exits $L < \infty$ such that
\begin{align}
\| v_t(x) - v_{t'}(x') \| \leq L ( \|x-x' \| + |t-t'|).
\end{align}
\end{assumption}
%
\begin{assumption}
\label{asmp:dissip}
For all $t \geq 0$, $v_t$ is dissipative, i.e. for all $x \in \R^d$.
\begin{align}
\langle x, v_t(x) \rangle \geq m \|x\|^2 -b
\end{align}
for some $m,b >0$.
\end{assumption}
%
\begin{assumption}
\label{asmp:stochgrad}
The estimator of the drift is unbiased, i.e.\ $\E[\hat{v}_t] = v_t$ for all $t \geq 0$, and its variance satisfies the following condition for some $\delta \in (0,1)$ and for all $t\geq 0$, $x \in \R^d$:
\begin{align}
\E[ \|\hat{v}_t(x) - v_t(x) \|^2] \leq 2 \delta(L^2 \|x\|^2 + B^2).
\end{align}
\end{assumption}
%
\begin{assumption}
\label{asmp:init_fun}
The function $\Psi_t$ satisfies the following conditions for all $t \geq 0$:
\begin{align}
|\Psi_t(0)| \leq A, \qquad \text{and} \qquad \|v_t(0)\| \leq B
\end{align}
for $A,B \geq 0$.
\end{assumption}


We are interested in computing the distance $\| \muh_{Kh} - \nu_\lambda \|_{\TV}$, where $\muh_{Kh}$ denotes the law of $\bar{X}_K$ with step size $h$. In order to upper-bound this distance, we follow the approach presented in \cite{dalalyan2017theoretical} and \cite{raginsky17a}, where we decompose the into two terms: $\| \muh_{Kh} - \nu_\lambda \|_{\TV} \leq \| \muh_{Kh} - \mu_T \|_{\TV} + \| \mu_{T} - \nu_\lambda \|_{\TV}$, where $\mu_T$ denotes the law of $X_T$ such that $T=Kh$. %$(X_t)_t$ is the solution of the continuous-time SDE \eqref{eqn:sde} and 

We start by upper-bounding the first term. 
%
\begin{lemma}
\label{lem:euler}
Assume that the conditions \Cref{asmp:lipschitz,asmp:stochgrad,asmp:dissip,asmp:init_fun} hold. Then, the following bound holds:
\begin{align}
\| \muh_{Kh} - \mu_{T} \|_{\TV}^2 \leq \frac{L^2 K}{4\lambda} \Bigl( \frac{C_1 h^3}{3} + 3 \lambda d h^2 \Bigr) + \frac{C_2 \delta K h}{8\lambda},
\end{align}
where the constants $C_1$ and $C_2$ are explicitly defined in the proof. 
\end{lemma}


\begin{thm}
\label{thm:euler}
Assume that \Cref{asmp:sde_ergo,asmp:sde_expconv,asmp:lipschitz,asmp:stochgrad,asmp:dissip,asmp:init_fun} hold. Then, the following bound holds:
\begin{align}
\| \muh_{Kh} - \nu_\lambda \|_{\TV} \leq \left \lbrace  \frac{L^2 K}{4\lambda} \Bigl( \frac{C_1 h^3}{3} + 3 \lambda d h^2 \Bigr) + \frac{C_2  \delta K h}{8\lambda} \right \rbrace^{1/2} +  C_3 \exp(-C_4 Kh \lambda),
\end{align}
for some $C_1,C_2,C_3,C_4 > 0$.
\end{thm}

\begin{cor}
  \label{coro:precision}
  Assume that \Cref{asmp:sde_ergo,asmp:sde_expconv,asmp:lipschitz,asmp:stochgrad,asmp:dissip,asmp:init_fun} hold. Then for all $\varepsilon >0$, setting
  \begin{align}
T = Kh  = \ceil{-\log(\varepsilon/(2C_3))/(C_4\lambda)} \, , \qquad 
h = (3/C_1)\wedge\left(\frac{\varepsilon^2 \lambda}{L^2 T}(1+3\lambda d)^{-1}\right)^{1/2} \,,
  \end{align}
  we have
  \begin{align}
    \| \muh_{Kh} - \nu_\lambda \|_{\TV} \leq \varepsilon + \left(\frac{C_2 \delta K h}{8\lambda}\right)^{1/2} . 
  \end{align}
\end{cor}
