%!TEX root = ./nips_2018_sketchmcmc.tex

\section{Preliminaries and Technical Background}
\label{sec:techbg}

% \begin{itemize}
% \item Brief intro to gradient flows in $\W$ 
% \item Fokker-Planck equations and relations to SDEs
% \item Langevin equation as a special case
% \item SGLD for optimization \cite{raginsky17a,zhang17b}
% \end{itemize}

% \subsection{Notation}

% For a vector $v \in \R^d$, we denote its $\ell_2$-norm by $\|v\| \triangleq \sqrt{v^\top v}$. We occasionally denote the Euclidean inner-product of two vectors by $\langle v, y \rangle \triangleq v^\top y$ for $v,y \in \R^d$. 

\subsection{Wasserstein distance, optimal transport maps, and Kantorovich potentials}


For two probability measures $\mu,\nu \in \PS(\Omega)$, the 2-Wasserstein distance is defined as follows:
\begin{align}
\W(\mu,\nu) \triangleq \Biggl\{ \inf_{\gamma \in {\cal C}(\mu,\nu)} \int_{\Omega \times \Omega} \|x-y\|^2 \gamma(dx , dy) \Biggr\}^{1/2}, \label{eqn:w2}
\end{align}
where ${\cal C}(\mu,\nu)$ is called the set of \emph{transportation plans} and denotes the space of all joint distributions whose marginals coincide with $\mu$ and $\nu$. More precisely, ${\cal C}(\mu,\nu) \triangleq \{\gamma \in \PS(\Omega \times \Omega) | \Pi_{1\#}\gamma = \mu, \Pi_{2\#}\gamma = \nu  \}$, where $\Pi_1(x,y) \triangleq x$ and $\Pi_2(x,y) \triangleq y$, $\Pi_{1\#}$ and $\Pi_{2\#}$ then denote the marginalization operators. 


Since $\W$ is a linear program with linear constraints, it has a dual formulation, known as the Kantorovich problem, defined as follows:
\begin{align}
\W(\mu,\nu) = \sup_{\psi \in \textnormal{1-Lipschitz}} \sqrt{2}\Bigl\{ \int_\Omega \psi(x) \mu(dx) + \int_\Omega \psi^c(x) \nu(dx) \Bigr\}^{1/2}, \label{eqn:w2dual}
\end{align}
where $\psi^c$ denotes the c-conjugate of $\psi$ and is defined as follows: $\psi^c(y) \triangleq \{ \inf_{x\in \Omega} \| x-y\|^2 - \psi(x)\}$. The functions $\psi$ that realize the supremum in \eqref{eqn:w2dual} are called the Kantorovich potentials between $\mu$ and $\nu$.
%
Provided that $\mu$ satisfies a mild regularity condition, we have the following nice uniqueness result.
\begin{thm}[Theorem 1.4 of \cite{santambrogio2010introduction}]
\label{thm:unqmap}
If $\mu$ is absolutely continuous with respect to the Lebesgue measure, then there exists a unique optimal transport plan $\gamma^\star$ that realizes the infimum in \eqref{eqn:w2} and it is of the form $(\text{Id} \times T)_\# \mu$. Furthermore, there exists at least a Kantorovich potential $\psi$ whose gradient $\nabla \psi$ is uniquely determined $\mu$-almost everywhere. The optimal transport map $T$ and the potential $\psi$ are linked by $T(x) = x- \nabla \psi(x)$.
\end{thm}
 

This result implies that there exists a perfect solution for transporting samples from $\mu$ to samples from $\nu$ and even do so in a way that minimizes the $\ell_2$ displacement. However, identifying this solution is highly non-trivial. In the discrete case, effective solutions have been proposed \cite{cuturi2013sinkhorn}. However, for continuous and high-dimensional probability measures, constructing an actual transport plan remains a challenge. Even if recent contributions \cite{genevay2016stochastic} have made it possible to rapidly compute $\W$, they do so without constructing the optimal map $T$, which is our objective here.


\subsection{Gradient flows in the Wasserstein space}

% \umut{I'll fill this place once Szymon writes down his parts.}

\begin{itemize}
\item Talk about generalized minimizing movements
\end{itemize}


\subsection{Sliced-Wasserstein distance}
\label{sec:sw}

In the one-dimensional case, i.e.\ $\mu,\nu \in \PS(\R)$, the Wasserstein distance has an analytical form, given as follows:
\begin{align}
\W(\mu,\nu) = \int_0^1 |F_\mu^{-1}(\tau) - F_\nu^{-1}(\tau)|^2 \> d\tau,
\end{align}
where $F_\mu$ and $F_\nu$ denote the cumulative distribution functions (CDF) of $\mu$ and $\nu$, respectively, and $F^{-1}_\mu, F^{-1}_\nu$ denote the inverse CDFs that can be interpreted as the \emph{quantile} functions.
%
In this case, the optimal transport map has a closed-form formula, given as follows: $T(x) = (F_\nu^{-1} \circ F_\mu) (x)$. The optimal map $T$ is also known as the \emph{increasing arrangement}, which maps each quantile of $\mu$ to the same quantile of $\nu$, e.g. minimum to minimum, median to median, maximum to maximum.
%
Due to Theorem~\ref{thm:unqmap}, the derivative of the corresponding Kantorovich potential is given as $\psi'(x) \triangleq \partial_x \psi(x) = x- (F_\nu^{-1} \circ F_\mu) (x)$.
 
In the multidimensional case $d > 1$, building a transport map is much more difficult. The nice properties of the one-dimensional Wasserstein distance motivate the usage of \emph{sliced-Wasserstein distance} ($\SW$) for practical applications. Before formally defining $\SW$, let us first define the orthogonal projection $\theta^* (x) \triangleq \langle \theta, x \rangle$ for any direction $\theta \in \Sp^{d-1}$ and $x \in \R^d$, where $\langle \cdot, \cdot \rangle$ denotes the Euclidean inner-product and $\Sp^{d-1} \subset \R^d$ denotes the $d$-dimensional unit sphere. Then, the $\SW$ distance is formally defined as follows:
\begin{align}
\SW(\mu,\nu) \triangleq \int_{\Sp^{d-1}} \W (\theta^*_\#\mu, \theta^*_\#\nu) \> d \theta, \label{eqn:sw}
\end{align}
where $d\theta$ represents the uniform probability measure on $\Sp^{d-1}$, such that $\int_{\Sp^{d-1}} d \theta = 1$. As shown \cite{bonnotte2013unidimensional}[Chapter 5], $\SW$ is indeed a distance metric and induces the same topology as $\W$ in compact domains. 

The $\SW$ has a strong practical significance: provided that the distributions $\theta^*_\#\mu$ and $\theta^*_\#\nu$ can be computed, then for any $\theta \in \Sp^{d-1}$, the distance $\W (\theta^*_\#\mu, \theta^*_\#\nu)$, as well as its optimal transport map and the corresponding Kantorovich potential can be analytically computed. Therefore, one can easily approximate \eqref{eqn:sw} by using a simple Monte Carlo scheme that draws uniform random samples from $\Sp^{d-1}$ and replaces the integral in \eqref{eqn:sw} with a finite-sample average. 





