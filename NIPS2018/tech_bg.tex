%!TEX root = ./nips_2018_sketchmcmc.tex

\section{Preliminaries and Technical Background}
\label{sec:techbg}

% \begin{itemize}
% \item Brief intro to gradient flows in $\W$ 
% \item Fokker-Planck equations and relations to SDEs
% \item Langevin equation as a special case
% \item SGLD for optimization \cite{raginsky17a,zhang17b}
% \end{itemize}

% \subsection{Notation}

% For a vector $v \in \R^d$, we denote its $\ell_2$-norm by $\|v\| \triangleq \sqrt{v^\top v}$. We occasionally denote the Euclidean inner-product of two vectors by $\langle v, y \rangle \triangleq v^\top y$ for $v,y \in \R^d$. 

\subsection{Optimal Transport, Wasserstein Spaces, Gradient Flows}


For two probability measures $\mu,\nu \in \PS(\Omega)$, the 2-Wasserstein distance is defined as follows:
\begin{align}
\W(\mu,\nu) \triangleq \Biggl\{ \inf_{\gamma \in {\cal C}(\mu,\nu)} \int_{\Omega \times \Omega} \|x-y\|^2 \gamma(dx , dy) \Biggr\}^{1/2}, \label{eqn:w2}
\end{align}
where ${\cal C}(\mu,\nu)$ is called the set of \emph{transportation plans} and denotes the space of all joint distributions whose marginals coincide with $\mu$ and $\nu$. More precisely, ${\cal C}(\mu,\nu) \triangleq \{\gamma \in \PS(\Omega \times \Omega) | \Pi_{1\#}\gamma = \mu, \Pi_{2\#}\gamma = \nu  \}$, where $\Pi_1(x,y) \triangleq x$ and $\Pi_2(x,y) \triangleq y$, $\Pi_{1\#}$ and $\Pi_{2\#}$ then denote the marginalization operators.

Since $\W$ is a linear program with linear constraints, it has a dual formulation, known as the Kantorovich problem:
\begin{align}
\W(\mu,\nu) = \max_{\psi \in \textnormal{1-Lipschitz}} \Bigl\{ \int_\Omega \psi(x) \mu(dx) + \int_\Omega \psi^c(x) \nu(dx) \Bigr\}, \label{eqn:w2dual}
\end{align}
where $\psi^c$ denotes the c-conjugate of $\psi$ and is defined as follows: $\psi^c(y) \triangleq \{ \inf_{x\in \Omega} \| x-y\|^2 - \psi(x)\}$. The functions $\psi$ that realize the maximum in \eqref{eqn:w2dual} are called the Kantorovich potentials between $\mu$ and $\nu$.









