%!TEX root = ./nips_2018_sketchmcmc_supp.tex

\section{The Gradient Flow and the SDE}

Let $\rho_t$ be the density of a measure $\mu_t$ with respect to the Lebesgue measure, such that $\mu_t(dx) = \rho_t(x) dx$. In this section, we will be interested in the following gradient flow in $\W$:
\begin{align}
\partial_t \rho_t &= - \nabla_{\W} \F_\lambda(\rho_t) \\
&=  \nabla \cdot (\rho_t \> v_t) + \lambda \Delta \rho_t, \label{eqn:gradflow_reg}
\end{align}
where 
\begin{align}
v_t(x) \triangleq \nabla \Psi_t(x) = \int_{\Sp^{d-1}} \psi'_{t,\theta}(\langle \theta, x \rangle) \theta \> d\theta \label{eqn:idt_v}
\end{align}
and
\begin{align}
\Psi_t(x) \triangleq \int_{\mathbb{S}^{d-1}} \psi_{t,\theta}(\langle \theta, x \rangle) \> d\theta.
\end{align}
Here, $\psi_{t,\theta}$ denotes the Kantorovich potential between $\theta^*_{\#}\mu_t$ and $\theta^*_{\#}\nu$ and $d\theta$ represents the uniform probability measure on $\Sp^{d-1}$, such that $\int_{\Sp^{d-1}} d \theta = 1$.

We now consider the modified flow given in \eqref{eqn:gradflow_reg}. We can observe that, this equation is the Fokker-Planck equation associated with the following stochastic differential equation (SDE):
\begin{align}
d X_t = - v_t(X_t) dt + \sqrt{2 \lambda } d W_t, \label{eqn:sde}
\end{align}
where $W_t$ denotes the standard Brownian motion.

\begin{assumption}
\label{asmp:sde_ergo}
The SDE in \eqref{eqn:sde} has a unique strong solution denoted by $(X_t)_{t\geq 0}$. The solution process has a unique invariant measure denoted by $\nu_\lambda$ and it is geometrically ergodic. 
\end{assumption}

We first provide a bound between the invariant measure $\nu_\lambda$ and $\nu$.

\begin{prop}
\label{prop:dist_statmeas}
Consider the following SDE
\begin{align}
d Y_t = - v_t(Y_t) dt + \sqrt{2 \epsilon } d W_t. \label{eqn:sde_eps}
\end{align}
Assume that it satisfies \Cref{asmp:sde_ergo} with the invariant measure denoted by $\nu_\epsilon$. Let $\rho^\epsilon_t$ denote the probability density function of $Y_t$ at time $t$. Further assume that for all $\epsilon,t>0$, there exists $C >0$  
\begin{align}
\int_{0}^t \int_{\R^d} \frac{\|\nabla \rho^\epsilon_s(x) \|^2}{\rho^\epsilon_s(x)} dx ds <C, \qquad \text{and} \qquad \int_{0}^t \int_{\R^d}  \frac{\|\nabla \rho^\epsilon_s(x)\|}{1+\|x\|} dx ds <\infty.
\end{align}
Then the following bound holds:
\begin{align}
\lim_{\epsilon \rightarrow 0} \| \nu_\lambda - \nu_\epsilon \|_{\TV}^2 \leq 2 C \lambda.
\end{align}
\end{prop}
%
\begin{proof}
By Corollary 1.2 of \cite{bogachev2016distances}, for any $\epsilon > 0$ we have 
\begin{align}
\| \nu_\lambda - \nu_\epsilon \|_{\TV}^2 &\leq \int_0^\infty \int_{\R^d} \Bigl| \Bigl(\frac{\sqrt{\lambda}}{\sqrt{\epsilon}}- \frac{\sqrt{\epsilon}}{\sqrt{\lambda}} \Bigr) \sqrt{2\epsilon}  \Bigr|^2 \frac{\|\nabla \rho^\epsilon_s(x)\|^2}{\rho^\epsilon_s(x)}  dx ds \\
&\leq  C \Bigl| \Bigl(\frac{\sqrt{\lambda}}{\sqrt{\epsilon}}- \frac{\sqrt{\epsilon}}{\sqrt{\lambda}} \Bigr) \sqrt{2\epsilon}  \Bigr|^2 \label{eqn:prop_interm} \\
&= \frac{2C}{\lambda} (\lambda - \epsilon)^2,
\end{align}
where \eqref{eqn:prop_interm} is obtained by the assumption. The desired result is obtained by taking the taking the limit of both sides. 
\end{proof}
%



\section{Euler Discretization}


Corollary~\ref{prop:dist_statmeas} shows that if we could simulate \eqref{eqn:sde}, then we could use the sample paths $(X_t)_t$ as
samples drawn from $\nu_\lambda$, which is not far from $\nu$. However, this is not possible since the drift $v_t$ cannot be computed analytically, and the SDE \eqref{eqn:sde} is a continuous-time process.

We now consider the approximate Euler-Maruyama discretization, that is given as follows:
\begin{align}
\bar{X}_{n+1} = \bar{X}_n - h \hat{v}_n(\bar{X}_n) + \sqrt{2 \lambda h} Z_{n+1},
\end{align}
where $\{Z_n\}_{n}$ denotes a series of standard Gaussian random variables, $h$ denotes the step-size, and $\hat{v}_n$ is a computable unbiased estimator of $v_n$.


By Theorem 5.6.1 of \cite{bonnotte2013unidimensional}, we know that $\Psi_t$ is Lipschitz continuous. We consider the following assumptions:
\begin{assumption}
\label{asmp:lipschitz}
The drift is Lipschitz continuous, i.e.\ there exits $L < \infty$ such that
\begin{align}
\| v_t(x) - v_{t'}(x') \| \leq L ( \|x-x' \| + |t-t'|).
\end{align}
\end{assumption}
%
\begin{assumption}
\label{asmp:dissip}
For all $t \geq 0$, $v_t$ is dissipative, such that
\begin{align}
\langle x, v_t \rangle \geq m \|x\|^2 -b
\end{align}
for some $m,b >0$.
\end{assumption}
%
\begin{assumption}
\label{asmp:stochgrad}
The estimator of the drift is unbiased, i.e.\ $\E[\hat{v}_t] = v_t$ for all $t \geq 0$, and its variance satisfies the following condition for some $\delta \in (0,1)$ and for all $t\geq 0$, $x \in \R^d$:
\begin{align}
\E[ \|\hat{v}_t(x) - v_t(x) \|^2] \leq 2 \delta(M \|x\|^2 + B^2).
\end{align}
\end{assumption}
%
\begin{assumption}
The function $\Psi_t$ satisfies the following conditions for all $t \geq 0$:
\begin{align}
|\Psi_t(0)| \leq A, \qquad \text{and} \qquad \|v_t(0)\| \leq B
\end{align}
for $A,B \geq 0$.
\end{assumption}

\section{Discussion on the assumptions}

\subsection{Conditions for the unique solution to the flow}

The following conditions ensure that there is a unique solution to the flow given in \eqref{eqn:gradflow_reg}:
\begin{assumption}
\label{asmp:flowunq_1}
There exists $p>d+2$ such that for every open ball $B \subset \R^d$, one has
\begin{align}
\int_0^T \int_B \|v_t(x)\|^p dx\> dt < \infty.
\end{align}
\end{assumption}
%
\begin{assumption}
\label{asmp:flowunq_2}
The initial measure $\mu_0$ has finite entropy, such that
\begin{align}
\int_{\R^d} \rho_0(x) \log \rho_0(x) dx < \infty,
\end{align}
where $\mu_0(dx) = \rho_0(x)dx$ and $\rho_0 \in L^1(\R^d)$.
\end{assumption}
%
\begin{assumption}
\label{asmp:flowunq_3}
There exist $\alpha, \gamma, \delta, c, k \in \R_+$ such that for all $(t,x) \in [0,T] \times \R^d$ 
\begin{align}
\langle v_t(x), x \rangle \leq \gamma - (ck + \delta) \| x\|^{2k},
\end{align}
and 
\begin{align}
\label{asmp:flowunq_4}
\|v_t(x)\| \leq \alpha \exp(\frac{c}{2} \|x\|^{2k}), \qquad \text{and} \qquad \int_{\R^d} \exp(\frac{c}{2}\|x\|^{2k} ) \mu_0(dx) < \infty.
\end{align}
\end{assumption}

The following theorem ensures the uniqueness.
\begin{thm}[Theorem 3.3 \cite{bogachev2007uniqueness}]
Assume that \Cref{asmp:flowunq_1,asmp:flowunq_2,asmp:flowunq_3,asmp:flowunq_4} hold. Then, there exists a unique family $\{\mu_t, t\in(0,T]\}$ of probability measures on $\R^d$ solving \eqref{eqn:gradflow_reg}.
\end{thm}

\subsection{The assumptions of Proposition~\ref{prop:dist_statmeas}}

The assumptions of Proposition~\ref{prop:dist_statmeas} can be satisfied if we further assume certain regularity assumptions on $v_t$. For more information and a discussion about these assumptions, we refer the reader to Remark 1.4 in \cite{bogachev2016distances} and \cite{bogachev2006global,bogachev2008estimates}.


