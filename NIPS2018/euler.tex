%!TEX root = ./nips_2018_sketchmcmc_supp.tex

% \section{The Gradient Flow and the SDE}

% Let $\rho_t$ be the density of a measure $\mu_t$ with respect to the Lebesgue measure, such that $\mu_t(dx) = \rho_t(x) dx$. In this section, we will be interested in the following gradient flow in $\W$:
% \begin{align}
% \partial_t \rho_t &= - \nabla_{\W} \F_\lambda(\rho_t) \\
% &=  \nabla \cdot (\rho_t \> v_t) + \lambda \Delta \rho_t, \label{eqn:gradflow_reg}
% \end{align}
% where 
% \begin{align}
% v_t(x) \triangleq \nabla \Psi_t(x) = \int_{\Sp^{d-1}} \psi'_{t,\theta}(\langle \theta, x \rangle) \theta \> d\theta \label{eqn:idt_v}
% \end{align}
% and
% \begin{align}
% \Psi_t(x) \triangleq \int_{\mathbb{S}^{d-1}} \psi_{t,\theta}(\langle \theta, x \rangle) \> d\theta.
% \end{align}
% Here, $\psi_{t,\theta}$ denotes the Kantorovich potential between $\theta^*_{\#}\mu_t$ and $\theta^*_{\#}\nu$ and $d\theta$ represents the uniform probability measure on $\Sp^{d-1}$, such that $\int_{\Sp^{d-1}} d \theta = 1$.

% We now consider the modified flow given in \eqref{eqn:gradflow_reg}. We can observe that, this equation is the Fokker-Planck equation associated with the following stochastic differential equation (SDE):
% \begin{align}
% d X_t = - v_t(X_t) dt + \sqrt{2 \lambda } d W_t, \label{eqn:sde}
% \end{align}
% where $W_t$ denotes the standard Brownian motion.

\section{Proof of Proposition~\ref{prop:dist_statmeas}}

%
\begin{proof}
By Corollary 1.2 of \cite{bogachev2016distances}, for any $\epsilon > 0$ we have 
\begin{align}
\| \nu_\lambda - \nu_\epsilon \|_{\TV}^2 &\leq \int_0^\infty \int_{\R^d} \Bigl| \Bigl(\frac{\sqrt{\lambda}}{\sqrt{\epsilon}}- \frac{\sqrt{\epsilon}}{\sqrt{\lambda}} \Bigr) \sqrt{2\epsilon}  \Bigr|^2 \frac{\|\nabla \rho^\epsilon_s(x)\|^2}{\rho^\epsilon_s(x)}  dx ds \\
&\leq  C \Bigl| \Bigl(\frac{\sqrt{\lambda}}{\sqrt{\epsilon}}- \frac{\sqrt{\epsilon}}{\sqrt{\lambda}} \Bigr) \sqrt{2\epsilon}  \Bigr|^2 \label{eqn:prop_interm} \\
&= \frac{2C}{\lambda} (\lambda - \epsilon)^2,
\end{align}
where \eqref{eqn:prop_interm} is obtained by the assumption. The desired result is obtained by taking the taking the limit of both sides. 
\end{proof}
%

\section{Proof of Lemma~\ref{lem:euler}}
%
%
\begin{proof}
We use the proof technique presented in \cite{dalalyan2017theoretical,raginsky17a}. Starting from the discrete-time process $(\bar{X}_k)_{k\in \mathbb{N}_+}$, we first define a continuous-time process $(Y_t)_{t\geq 0}$ that linearly interpolates $(\bar{X}_k)_{k\in \mathbb{N}_+}$, given as follows: 
\begin{align}
d Y_t = \tilde{v}_t(Y) dt + \sqrt{2 \lambda} dW_t, \label{eqn:sde_linear}
\end{align}
where $\tilde{v}_t(Y) \triangleq - \sum_{k=0}^{\infty} \hat{v}_k (Y_{kh}) \mathds{1}_{[kh, (k+1)h)}(t)$ and $\mathds{1}$ denotes the indicator function. It is easy to verify that for all $k \in \mathbb{N}_+$, we have $Y_{kh} = \bar{X}_k$. 

Let us denote the distributions of $(X_t)_{t \in [0,T]}$ and $(Y_t)_{t \in [0,T]}$ as $\pi_{X}^T$ and $\pi_{Y}^T$ with $T = Kh$. Then we can use Girsanov's formula to express the Kullback-Leibler (KL) divergence between these two distributions, given as follows:
\begin{align}
\KL (\pi_{X}^T || \pi_{Y}^T) &= \frac1{4 \lambda} \int_0^{Kh} \E[ \|v_t(Y_t) + \tilde{v}_t(Y) \|^2 ]  \> dt \\
&= \frac1{4 \lambda} \sum_{k=0}^{K-1} \int_{kh}^{(k+1)h} \E[ \|v_t(Y_t) + \tilde{v}_t(Y) \|^2 ] \> dt \\
&= \frac1{4 \lambda} \sum_{k=0}^{K-1} \int_{kh}^{(k+1)h} \E[ \|v_t(Y_t) - \hat{v}_{kh}(Y_{kh}) \|^2 ] \> dt,
\end{align}
where we use the notation $\hat{v}_{kh} \equiv \hat{v}_{k}$ in order to illustrate the time index more explicitly. By using $v_t(Y_t) - \hat{v}_{kh}(Y_{kh}) = ( v_t(Y_t) - v_{kh}(Y_{kh})) + ( v_{kh}(Y_{kh}) - \hat{v}_{kh}(Y_{kh}))$, we obtain
%
\begin{align}
\nonumber \KL (\pi_{X}^T || \pi_{Y}^T) \leq& \frac1{2 \lambda} \sum_{k=0}^{K-1} \int_{kh}^{(k+1)h} \E[ \|v_t(Y_t) - {v}_{kh}(Y_{kh}) \|^2 ] \> dt \\
&+  \frac1{2 \lambda} \sum_{k=0}^{K-1} \int_{kh}^{(k+1)h} \E[ \|v_{kh}(Y_{kh}) - \hat{v}_{kh}(Y_{kh}) \|^2 ] \> dt \\
\nonumber \leq& \frac{L^2}{\lambda} \sum_{k=0}^{K-1} \int_{kh}^{(k+1)h} \bigl(\E[ \|Y_t - Y_{kh} \|^2 ] + (t-kh)^2 \bigr)  \> dt \\
&+  \frac1{2 \lambda} \sum_{k=0}^{K-1} \int_{kh}^{(k+1)h} \E[ \|v_{kh}(Y_{kh}) - \hat{v}_{kh}(Y_{kh}) \|^2 ] \> dt . \label{eqn:lem1_proof_interm}
\end{align}
The last inequality is due to the Lipschitz condition \Cref{asmp:lipschitz}.

Now, let us focus on the term $\E[ \|Y_t - Y_{kh} \|^2]$. By using \eqref{eqn:sde_linear}, we obtain:
\begin{align}
Y_t - Y_{kh} = - (t-kh) \hat{v}_{kh}(Y_{kh}) + \sqrt{2 \lambda (t-kh)} Z,
\end{align}
where $Z$ denotes a standard normal random variable. By adding and subtracting the term $-(t-kh) v_{kh}(Y_{kh})$, we have:
\begin{align}
Y_t - Y_{kh} = -(t-kh)v_{kh}(Y_{kh}) + (t-kh)(v_{kh}(Y_{kh}) - \hat{v}_{kh}(Y_{kh})) + \sqrt{2 \lambda (t-kh)} Z.
\end{align}
Taking the square and then the expectation of both sides yields:
\begin{align}
\nonumber \E[ \|Y_t - Y_{kh} \|^2] \leq& 3(t-kh)^2 \E[ \|v_{kh}(Y_{kh})\|^2] + 3 (t-kh)^2 \E[\|v_{kh}(Y_{kh}) - \hat{v}_{kh}(Y_{kh})\|^2] \\
&+ 6\lambda (t-kh)d.
\end{align}
As a consequence of \Cref{asmp:lipschitz} and \Cref{asmp:init_fun}, we have $\| v_t(x)\| \leq L\|x\|+B$ for all $t \geq 0$, $x\in \R^d$. Combining this inequality with \Cref{asmp:stochgrad}, we obtain:
\begin{align}
\nonumber \E[ \|Y_t - Y_{kh} \|^2] \leq& 6(t-kh)^2 (L^2 \E[ \|Y_{kh}\|^2] + B^2) + 6(t-kh)^2 (L^2 \E[ \|Y_{kh}\|^2] + B^2) \\
&+ 6\lambda (t-kh)d\\
=& 12(t-kh)^2 (L^2 \E[ \|Y_{kh}\|^2] + B^2) + 6\lambda (t-kh)d.
\end{align}
By Lemma 3.2 of \cite{raginsky17a}, we have $\E[ \|Y_{kh}\|^2] \leq C_0 \triangleq C_e +2  (1 \vee \frac1{m})(b+2B^2 + d \lambda)$, where $C_e$ denotes the entropy of $\mu_0$. Using this result in the above equation yields:
\begin{align}
\E[ \|Y_t - Y_{kh} \|^2] \leq& 12(t-kh)^2 (L^2 C_0 + B^2) + 6\lambda (t-kh)d. \label{eqn:lem_bound1}
\end{align}

We now focus on the term $\E[ \|v_{kh}(Y_{kh}) - \hat{v}_{kh}(Y_{kh}) \|^2 ]$ in \eqref{eqn:lem1_proof_interm}. Similarly to the previous term, we can upper-bound this term as follows:
\begin{align}
\E[ \|v_{kh}(Y_{kh}) - \hat{v}_{kh}(Y_{kh}) \|^2 ] \leq& 2 \delta(L^2 \E[\|Y_{kh}\|^2] + B^2) \\
\leq& 2 \delta(L^2 C_0 + B^2). \label{eqn:lem_bound2}
\end{align}

By using \eqref{eqn:lem_bound1} and \eqref{eqn:lem_bound2} in \eqref{eqn:lem1_proof_interm}, we obtain:
\begin{align}
\nonumber \KL (\pi_{X}^T || \pi_{Y}^T) \leq& \frac{L^2}{\lambda} \sum_{k=0}^{K-1} \int_{kh}^{(k+1)h} \bigl(12(t-kh)^2 (L^2 C_0 + B^2) + 6\lambda (t-kh)d +(t-kh)^2 \bigr) dt\\
&+  \frac1{2 \lambda} \sum_{k=0}^{K-1} \int_{kh}^{(k+1)h} 2 \delta(L^2 C_0 + B^2) \> dt \\
=& \frac{L^2 K}{\lambda} \Bigl( \frac{C_1 h^3}{3} + \frac{6 \lambda d h^2}{2} \Bigr) + \frac{C_2 \delta K h}{2\lambda},
\end{align}
where $C_1 = 12(L^2 C_0 + B^2)+1$ and $C_2 = 2 (L^2 C_0 + B^2)$.

Finally, by using the data processing and Pinsker inequalities, we obtain:
\begin{align}
\| \muh_{Kh} - \mu_{T} \|_{\TV}^2 \leq \| \pi_{X}^T - \pi_{Y}^T \|_{\TV}^2 \leq& \frac1{4} \KL (\pi_{X}^T || \pi_{Y}^T) \\
=& \frac{L^2 K}{4\lambda} \Bigl( \frac{C_1 h^3}{3} + 3 \lambda d h^2 \Bigr) + \frac{C_2 \delta K h}{8\lambda}.
\end{align}
This concludes the proof.

\end{proof}


\section{Proof of Theorem~\ref{thm:euler}}
%
\begin{proof}
The proof is a direct consequence of the triangle inequality, Lemma~\ref{lem:euler}, and assumption \Cref{asmp:sde_expconv}.
\end{proof}

\section{Proof of Corollary~\ref{coro:precision}}
\begin{proof}
Considering the bound given in Lemma~\ref{lem:euler}, the choices of $T$ and $h$ imply that
\begin{align}
\frac{L^2 K}{4\lambda} \Bigl( \frac{C_1 h^3}{3} + \frac{6 \lambda d h^2}{2} \Bigr) \leq \frac{\varepsilon^2}{4}, \quad \text{and} \quad C_3 \exp(-C_4 Kh \lambda) \leq \frac{\varepsilon}{2}. \label{eqn:cor_Th}
\end{align}
By combining \eqref{eqn:cor_Th} with the inequality $\sqrt{a+b}\leq \sqrt{a} + \sqrt{b}$, we obtain the desired result. 
\end{proof}

\section{Discussion on the assumptions}

\umut{This section will be rewritten.}

\subsection{Conditions for the unique solution to the flow}

The following conditions ensure that there is a unique solution to the flow given in \eqref{eqn:gradflow_reg}:
\begin{assumption}
\label{asmp:flowunq_1}
There exists $p>d+2$ such that for every open ball $B \subset \R^d$, one has
\begin{align}
\int_0^T \int_B \|v_t(x)\|^p dx\> dt < \infty.
\end{align}
\end{assumption}
%
\begin{assumption}
\label{asmp:flowunq_2}
The initial measure $\mu_0$ has finite entropy, such that
\begin{align}
\int_{\R^d} \rho_0(x) \log \rho_0(x) dx < \infty,
\end{align}
where $\mu_0(dx) = \rho_0(x)dx$ and $\rho_0 \in L^1(\R^d)$.
\end{assumption}
%
\begin{assumption}
\label{asmp:flowunq_3}
There exist $\alpha, \gamma, \delta, c, k \in \R_+$ such that for all $(t,x) \in [0,T] \times \R^d$ 
\begin{align}
\langle v_t(x), x \rangle \leq \gamma - (ck + \delta) \| x\|^{2k},
\end{align}
and 
\begin{align}
\label{asmp:flowunq_4}
\|v_t(x)\| \leq \alpha \exp(\frac{c}{2} \|x\|^{2k}), \qquad \text{and} \qquad \int_{\R^d} \exp(\frac{c}{2}\|x\|^{2k} ) \mu_0(dx) < \infty.
\end{align}
\end{assumption}

The following theorem ensures the uniqueness.
\begin{thm}[Theorem 3.3 \cite{bogachev2007uniqueness}]
Assume that \Cref{asmp:flowunq_1,asmp:flowunq_2,asmp:flowunq_3,asmp:flowunq_4} hold. Then, there exists a unique family $\{\mu_t, t\in(0,T]\}$ of probability measures on $\R^d$ solving \eqref{eqn:gradflow_reg}.
\end{thm}

\subsection{The assumptions of Proposition~\ref{prop:dist_statmeas}}

The assumptions of Proposition~\ref{prop:dist_statmeas} can be satisfied if we further assume certain regularity assumptions on $v_t$. For more information and a discussion about these assumptions, we refer the reader to Remark 1.4 in \cite{bogachev2016distances} and \cite{bogachev2006global,bogachev2008estimates}.



%%% Local Variables:
%%% mode: latex
%%% TeX-master: "nips_2018_sketchmcmc_supp"
%%% End:
