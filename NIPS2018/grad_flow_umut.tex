%!TEX root = ./nips_2018_sketchmcmc_supp.tex

\section{Gradient flows info -- to be merged with the next section}

In order to prove that the path $(\rho_t)_t$ solves a PDE for a given gradient flow, one first needs to show that there exits a path $(\rho_t)_t$ that appears as the limit of the solution of a \emph{time-discretized} problem \cite{jordan1998variational,santambrogio2017euclidean}. In particular, we need to show that for $\mu_t(dx) = \rho_t(x)dx$, $(\mu_t)_{t\geq 0}$ is point-wise limit of a family $(\mu_t^h)_{t\geq 0}$, where $\mu_t^h = \mu_{kh}^h$ for $t \in [kh, (k+1)h)$ (i.e.\ piece-wise constant), and $\mu_{(k+1)h}^h$ solves the following optimization problem:
\begin{align}
\mu^{h}_{(k+1)h} = \argmin_\mu \mathcal{G}(h, k , \mu, \mu^h_{kh}),  %\triangleq \mathcal{F}(\mu) + \frac{1}{2h}\W(\mu, \mu^k) %\argmin_\rho \Bigl\{ \F(\rho) + \frac1{2h}\W^2(\rho, \rho_k^h) \Bigr\},
\end{align} 
where,
\begin{align}
\mathcal{G}(h, k , \mu_+, \mu_-) \triangleq \mathcal{F}(\mu_+) + \frac{1}{2h}\W(\mu_+, \mu_-).
\end{align}
If $(\mu_t)_{t\geq 0}$ appears as the limiting curve $\lim_{h\rightarrow 0}(\mu_t^h)_{t\geq 0}$, then $(\mu_t)_{t\geq 0}$ is called a \emph{generalized minimizing movement} \cite{santambrogio2017euclidean,bonnotte2013unidimensional}. Proving that $(\mu_t)_{t\geq 0}$ is a minimizing movement ensures that the gradient flow $\partial_t \mu_t = - \nabla_{\W} \F(\mu_t)$ exists; however, the form of the vector field $v$ in \eqref{eqn:pde} can only be determined by further analysis.